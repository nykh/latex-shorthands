\usepackage{amsmath}
\usepackage{amsfonts, amssymb}
\usepackage{amsthm} % for proof environment
\usepackage{mathtools} % for paired delimiter and DeclareMathOperator
\usepackage{amssymb} % \because and \therefore
\usepackage{booktabs} % \toprule, \midrule, \bottomrule

% Proof environments
\newtheorem{theorem}[section]{Theorem}
\newtheorem{definition}[section]{Definition}
\newtheorem{corollary}[section]{Corrolary}
\newtheorem{lemma}[section]{Lemma}
\newtheorem{claim}[section]{Claim}

\renewcommand{\qedsymbol}{$\blacksquare$}
\newcommand{\suchThat}{\text{, such that }}
\newcommand{\where}[1]{\text{, where #1}}
\newcommand{\tand}{\text{ and }}
\newcommand{\tor}{\text{ or }}
\newcommand{\Let}{\text{Let }}

% ceil and floor
\DeclarePairedDelimiter{\ceil}{\lceil}{\rceil}
\DeclarePairedDelimiter{\floor}{\lfloor}{\rfloor}

% optimiztion
\DeclareMathOperator*{\argmin}{arg min}
\DeclareMathOperator*{\argmax}{arg max}
\DeclareMathOperator{\sign}{sign}

% set theory
\DeclarePairedDelimiter{\set}{\{}{\}}
\DeclarePairedDelimiter{\Set}{\bigg\{}{\bigg\}}
\newcommand{\ind}[1]{\mathbf{1}\{~#1~\}}  % Indicator function
\renewcommand{\c}{^{\sf c}}

% fraction
\newcommand{\rec}[1]{\frac{1}{#1}}
\newcommand{\half}[1]{\frac{#1}{2}}

% two cases
\newcommand{\twocases}[4]{\begin{cases}#1&#2\\#3&#4\end{cases}}
\newcommand{\otherwise}{\text{otherwise}}

% calculus
\newcommand{\D}{~\textrm{d}}
\newcommand{\diff}[2]{\dfrac{\text{d}}{\text{d}#2} #1}
\newcommand{\diffs}[2]{\dfrac{\text{d}#1}{\text{d}#2}}
\newcommand{\pdiff}[2]{\dfrac{\partial}{\partial{#2}} #1}
\newcommand{\pdiffs}[2]{\dfrac{\partial #1}{\partial{#2}}}
\newcommand{\ppdiff}[3]{\dfrac{\partial^2 #1}{\partial{#2}\partial{#3}}}

% Scientific notations
\newcommand{\e}[1]{\ensuremath{\times 10^{#1}}}

% Statistical learning
\newcommand{\yhat}{\hat{Y}}
\DeclarePairedDelimiter{\dataset}{\langle}{\rangle}
\newcommand{\idx}[1]{^{(#1)}} % for the index into sample
\DeclareMathOperator{\softmax}{softmax}

% commonly used number sets
\newcommand{\Real}{\mathbb{R}}
\newcommand{\Nat}{\mathbb{N}}
\newcommand{\Rational}{\mathbb{Q}}
\newcommand{\Compwlex}{\mathbb{C}}
\newcommand{\Integer}{\mathbb{Z}}

% linear algebra
\newcommand{\T}{^{\sf T}}
\DeclareMathOperator{\diag}{diag}
\DeclareMathOperator{\tr}{tr}
\newcommand{\norm}[2][]{\lVert#2\rVert_{#1}}
\newcommand{\inner}[1]{{#1}\T #1}
\newcommand{\outter}[1]{#1 {#1}\T}
\newcommand{\inv}[1]{{#1}^{-1}}
\newcommand{\pinv}[1]{{#1}^+}

\newcommand{\quadratic}[2]{{#1}\T #2 #1}
\newcommand{\vect}[1]{\left(\begin{matrix}#1\end{matrix}\right)}
\newcommand{\mat}[1]{\left[\begin{matrix}#1\end{matrix}\right]}

% big O notation
\DeclareMathOperator{\bigO}{\mathcal{O}}
